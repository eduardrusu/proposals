% Template for Subaru proposals (as of S11B)
%
% Please do not break/join the \tagname line 
%
% e.g.,
% GOOD
% \PIfirstname {Taro}
% \PImidinitial{}
% \PIlastname  {Yamada}
%
% NG
% \PIfirstname{Taro} \PImidinitial{} \PIlastname{Yamada}
%  !!!  joining \tagname line !!!
%
% GOOD
% \target{Valencia CF}{03 00 00.0}{+27 00 00.0}{J2000.0}{$m_{B}<30.0(AB)$}
%
% NG
% \target{Valencia CF}{03 00 00.0}{+27 00 00.0}
% {J2000.0}{$m_{B}<30.0(AB)$}
%  !!! splitting \tagname line !!!
%%
% Version 6.07 2009.07.29  by Y.T.(delete hidetargets,starburst gals)
% Version 6.06 2008.08.03  by Y.T.
% Version 6.05 2008.02.05  by Y.T.
% Version 6.04 2007.07.30  by Y.T. ("deep survey" deleted)
% Version 6.03 2006.08.04  by Y.T.
% Version 6.02 2006.02.10  by Y.T.
% Version 6.01 2005.08.12  by Y.T.
% Version 6.00 2005.03.03  by RO
% Version 5.03 2004.10.01&10.06  by RO
% Version      2004.09.24  by Y.T.
% Version      2004.08.28  by RO
% Version      2004.03.10
%
% If you are using LaTeX2e, you should uncomment the 
% \documentclass and \usepackage lines and comment the 
% \documentstyle line.
%\documentclass{article}
%\usepackage{subaru}
\documentstyle[subaru]{article}
\begin{document}
% Set this to the current Subaru semester
\semester{S11B}

%%%%%%%%%% Proposal ID %%%%%%%%%%
% You can leave this entry simply blank at the time of 
% first submission. In case you revise the already 
% submitted proposal, please fill-in the assigned 
% proposal ID here.
\proposalid{}

%%%%%%%%%% 1.(+ 7.) Title of Proposal %%%%%%%%%%
% Enter your title here; it should fit on one line 
% when printed
%\title{LGS-AO imaging of Selected Anomalous Gravitationally Lensed Quasars}
\title{Subaru Telescope LGS-AO imaging of Gravitationally Lensed Quasars}
%%%%%%%%%% 2. Principal Investigator %%%%%%%%%%
% Enter the Principal Investigator's information here
% Initial of middle name may be given in \PImidinitial
% If an underscore ("_") is included in the e-mail address,  
% a backslash ("\") should be inserted just before it 
% (such as "\_") in order to avoid TeX compiling error.
\PIfirstname {Cristian Eduard}
\PImidinitial{}
\PIlastname  {Rusu}
\PIinstitute {University of Tokyo, National Astronomical Observatory of Japan}
\PIaddress   {2-21-1 Osawa, Mitaka, Tokyo 181-8588, JAPAN}
\PIemail     {eduard.rusu@nao.ac.jp}
\PIphone     {+81(0)90-6248-1504}
\PIfax       {+81422343528}

%%%%%%%%%% 3. Scientific Category %%%%%%%%%%
% Uncomment ONE of the following lines to indicate 
% the scientific category
%\SolarSystem
%\NormalStars
%\ExtrasolarPlanets
%\StarandPlanetFormation
%\CompactObjectsandSNe
%\MilkyWay
%\LocalGroup
%\ISM
%\NearbyGalaxies
%\AGNandQSOActivity
%\QSOAbsorptionLinesandIGM
%\ClustersofGalaxies
%\LargeScaleStructure
\GravitationalLenses
%\HighzGalaxies
%\CosmologicalParameters
%\Miscellaneous

%%%%%%%%%% 4. Abstract %%%%%%%%%%
% Enter abstract here. Please ensure it fits 
% in the space provided.
\begin{abstract}
We propose high-resolution near-infrared imaging of gravitationally
lensed quasars using Subaru Telescope laser guide star adaptive optics
(LGS-AO). We draw our lens sample from the SDSS Quasar Lens Search
(SQLS), the largest current quasar lens survey whose selection
function is understood well. The current SQLS sample comprises of $\sim$ 60
quasar lenses, only 22 of which have received high-resolution imaging
to date; thus our proposed LGS-AO imaging of 36 quasar lenses (21 proposed for S11B)
significantly expands a homogenous sample of lenses with available
high-resolution images.  The high-resolution imaging enabled by the
LGS-AO is essential for turning the SQLS quasar lenses into a useful
astrophysical tool, as it opens up many exciting scientific
applications including the structure and evolution of early-type
galaxies and the black hole to bulge mass
relation through lensed quasar host galaxies.  

A proposal with the same title has been accepted for GT observations. The final targets and scientific focus of the two proposals are different, therefore there is no overlap. As described in the Scientific Justification, for the S11B GT observations we focus on individual objects with peculiar morphology, whereas here we aim for ensemble studies.
\end{abstract}

%%%%%%%%%% 5. Co-Investigators %%%%%%%%%%
% Enter name and institution of each Co-I
% e.g., \CoI{I. Newton}{University of Cambridge}
% Please note that information of "individual persons"
% should be given here, which can not be replaced by the name 
% of a group or a team. You can append the "\CoI" field 
% as many as you like if necessary. 
\begin{investigators}
\CoI{Masanori Iye}{NAOJ}
\CoI{Masamune Oguri}{NAOJ}
\CoI{Yosuke Minowa}{NAOJ}
\CoI{Shin Oya}{NAOJ}
\CoI{Naohisa Inada}{RIKEN}
\end{investigators}

%%%%%%%%%% 6. List of Applicants' Related Publications %%%%%
% List all relevant publications here.
\begin{publications}

% 1. ``SDSS J133401.39+331534.3: A New Subarcsecond Gravitationally Lensed Quasar''\\ \hspace*{0.4cm} Rusu, C. E., Oguri, M., et al. 2011, ApJ, {\bf 738}, 30 \\ 2. ``The Sloan Digital Sky Survey Quasar Lens Search. I. Candidate Selection Algorithm''\\ \hspace*{0.4cm} Oguri, M., Inada, N., et al. 2006, AJ, {\bf 132}, 999  \\3. ``The Sloan Digital Sky Survey Quasar Lens Search. II. Statistical Lens Sample from the Third Data Release''\\ \hspace*{0.4cm} Inada, N., Oguri, M., et al. 2008, AJ, {\bf 135}, 496  \\4. ``The Sloan Digital Sky Survey Quasar Lens Search. III. Constraints on Dark Energy from the DR3 Lens Catalog''\\ \hspace*{0.4cm} Oguri, M., Inada, N., et al. 2008, AJ, {\bf 135}, 512  \\5. ``The Sloan Digital Sky Survey Quasar Lens Search. IV. Statistical lens sample from the fifth data release''\\ \hspace*{0.4cm} Inada, N., Oguri, M., et al. 2010, AJ, {\bf 140}, 403  \\6. ``Eight New Quasar Lenses from the Sloan Digital Sky Survey Quasar Lens Search''\\ \hspace*{0.4cm} Kayo, I., Inada, N., Oguri, M. et al. 2010, AJ,  139, 1614  \\7. ``Astronomy with Laser Guide Star Adaptive Optics'' \hspace*{0.4cm} Iye, M. 2009, Non Linear Optics, NBF6, OSA  \\8. ``A galaxy at a redshift 6.96'' \hspace*{0.4cm} Iye, M. et al. 2006, Nature, {\bf 443}, 186-188 

1. ``The Sloan Digital Sky Survey Quasar Lens Search. I. Candidate Selection
 Algorithm''\\
\hspace*{0.4cm} Oguri, M., Inada, N., et al. 2006, AJ, {\bf 132}, 999

2. ``The Sloan Digital Sky Survey Quasar Lens Search. II. Statistical Lens Sample from the Third Data Release''\\
\hspace*{0.4cm} Inada, N., Oguri, M., et al. 2008, 
AJ, {\bf 135}, 496

3. ``The Sloan Digital Sky Survey Quasar Lens Search. III. Constraints
   on Dark Energy from the DR3 Lens Catalog''\\
\hspace*{0.4cm} Oguri, M., Inada, N., et al. 2008, 
AJ, {\bf 135}, 512

4. ``Eight New Quasar Lenses from the Sloan Digital Sky Survey Quasar Lens
Search''\\
\hspace*{0.4cm} Kayo, I., Inada, N., Oguri, M. et al. 2010, 
AJ,  139, 1614 

5. ``Astronomy with Laser Guide Star Adaptive Optics''\\
\hspace*{0.4cm} Iye, M. 2009, Non Linear Optics, NBF6, OSA

6. ``A galaxy at a redshift 6.96''\\
\hspace*{0.4cm} Iye, M. et al. 2006, Nature, {\bf 443}, 186-188
\end{publications}


%%%%%%%%%% 8. Observing Run %%%%%%%%%%
% For each observing run, enter the instrument, 
% number of nights requested, lunar phase (Dark/Gray/Bright), 
% preferred dates, acceptable dates, and configuration/mode.
% Here are some examples: 
% \run{S-Cam}{3}{Dark}{Dec/early Jan}{Nov--Feb}{}
% or
% \run{IRCS+AO188}{1}{Bright/Gray}{Sep--Oct}{Aug--Nov}{Grism,LGS}
% or
% \run{MOIRCS}{2}{Bright}{Dec}{Nov--Jan}{MOS}
%
% !!! -------- To proposers of Intensive Program --------
% !!! Please indicate here the TOTAL number of requested nights
% !!! for the whole project (maybe over 2--4 semesters): that is, 
% !!! NOT the number of requested nights only for this semester.
%
\begin{observingrun}
\run{IRCS+LGS-AO}{2}{Dark/Gray}{August, January}{August, January}{Imaging}
\run{}{}{}{}{}{}

% Enter the minimum acceptable number of nights to achieve 
% your science goals. You may leave this empty if you require 
% your full request.
\minnights{0.5}

\end{observingrun}

%%%%%%%%%% 9. List of Targets %%%%%%%%%%
% Enter your targets here: name, RA, dec, equinox, magnitude 
% e.g., 
% \target{4C~41.17}{06 50 52.10}{+41 30 30.5}{J2000.0}{$K=19.1$}
\begin{targets}
   \target{SDSSJ0743+2457}{07 43 52.62}{+24 57 43.6}{J2000}{SDSS $r=19.01$}
   \target{SDSSJ0746+4403}{07 46 53.04}{+44 03 51.4}{J2000}{SDSS $r=18.67$}
   \target{SDSSJ0819+5356}{08 19 59.81}{+53 56 24.2}{J2000}{SDSS $r=17.64$}
   \target{SDSSJ0904+1512}{09 04 04.15}{+15 12 54.5}{J2000}{SDSS $r=17.65$}	\target{SDSSJ0924+0219}{09 24 55.80}{+02 19 25.0}{J2000}{SDSS $r=17.89$}
	\target{SDSSJ1001+5027}{10 01 28.61}{+50 27 56.9}{J2000}{SDSS $r=17.55$}
\target{SDSSJ1002+4449}{10 02 29.47}{+44 49 42.8}{J2000}{SDSS $r=18.27$}
    \target{SDSSJ1054+2733}{10 54 40.83}{+27 33 06.4}{J2000}{SDSS $r=16.83$}
     \target{SDSSJ1131+1915}{11 31 57.72}{+19 15 27.7}{J2000}{SDSS $r=17.95$}
       \target{SDSSJ1313+5151}{13 13 39.99}{+51 51 28.4}{J2000}{SDSS $r=17.87$}
     \target{SDSSJ1322+1052}{13 22 36.42}{+10 52 39.4}{J2000}{SDSS $r=18.53$}
      \target{SDSSJ1335+0118}{13 35 34.79}{+01 18 05.5}{J2000}{SDSS $r=17.92$}

\end{targets}

% If the available space here is insufficient, please use 
% "\moretargets" environment in "19. List of (more) targets", 
% which you find near to the end of this file.

%%%%%%%%%% 10. Scheduling Requirements %%%%%%%%%%
\begin{schedule}
% Explain any scheduling requests noted above.
%
% !!! *** To proposers of Bright/Gray nights observations ***
% !!! When one particular source or several sources whose       
% !!! coordinates are concentrated to a narrow region are 
% !!! planned to observe in bright or grey nights, 
% !!! observations may be severely affected by the Moon in some
% !!! particular nights. In such cases, those inconvenient 
% !!! or unacceptable dates should be explicitly indicated here.
%
\scheduling{
We prefer dark nights because of our use of faint tip-tilt stars for
the laser guide star observations. Gray nights are also acceptable.
In order to ensure the visibility of our targets, we request observations very early (August, three targets) and very late in the semester (January, the rest of the targets).}

% Uncomment the following line if you want to make observations 
% remotely from the Subaru office at Hilo. 
%\remoteobs
\end{schedule}

%%%%%%%%%% 11. Instrument Requirements %%%%%%%%%%
% Add any further instrumentation requirements, or details 
% of your own instrument.
% !!! If you are planning to make observations in the MOS   !!!
% !!! mode of MOIRCS, the number of required masks must be  !!!
% !!! clearly specified as the first thing in this entry.   !!!
% !!! Similarly, in case of S-Cam observations, the filter  !!!
% !!! set you are going to use must be fully described here.!!!
%
\instruments{}

%%%%%%%%%% 12. Experience %%%%%%%%%%
% Please briefly describe your experience, ability, need of 
% assistance, etc. for making observations with Subaru
\experience{
Three of us (M.~I., Y.~M., and S.~O.) have extensive experience
of extra galactic observations using Subaru NGS, and are also developing the
laser guide star system for Subaru. Two of us (M.~O. and N.~I.) are
PIs of the SDSS Quasar Lens Search, and have extensive experience
of analyzing quasar lens systems. 
}

%%%%%%%%%% 13. Backup Proposal in Poor Conditions %%%%%%%%%%
% Briefly describe your backup proposal, while including 
% specific names of the targets. In this case, please do not 
% forget to give the data of all these backup targets also in 
% "20. List of Backup Targets", so that we may include them
% in our database system similarly to those of main targets.
\backup{
In case the LGS mode is not available, we request NGS mode observations of SDSSJ1322+1052 (the only target available to NGS), as well as normal mode (no adaptive optics) observations of the large separation targets ($>2.5''$). These targets are specified in section 14.
}

%%%%%%%%%% 14. Observing Method and Technical Details %%%%%%%%%%
% Describe the observing method and technical details
% of your proposed observations. Please explicitly state
% the instrument configuration (filters, grisms, slit width,
% readout mode), intended exposure time, and required 
% sensitivity to achieve your scientific goals.
% !!!!! Specific Remark to Proposers of AO Observations !!!!!
% If you propose AO observations, please describe the nature 
% of the target (extended or point source) as well as the guide 
% star properties (separation, brightness, acceptable minimum 
% Strehl ratio), so that our support scientists can judge 
% the feasibility of observations.
\begin{technicalinfo}
\medskip
\vspace{-0.5cm}
We request reasonably deep IRCS K band imaging of $\sim20$ min or more, which including overhead will require $\sim45$ min per target.

In the Table we summarize the tip-tilt
stars we will use. SDSSJ1620+1203, SDSSJ1650+4251 and SDSSJ2343-0050 are visible in August, whereas the rest of the targets are best visible at the end of January.
\\




\begin{tabular}{lcccccclcccccc}
   \hline\hline
Name & $N_{\rm img}$ & $\theta$ & mag & dist & Name & $N_{\rm img}$ & $\theta$ & mag & dist  \\
     \hline
SDSSJ0743+2457 & 2 & 1.05 & $R=16.91$ & $23.4''$ & SDSSJ1335+0118 & 2 & 1.57 & $R=17.55$ & $<5''$\\
SDSSJ0746+4403 & 2 & 1.08 & $R=15.45$ & $31.6''$ & SDSSJ1335+0527 & 2 & $\sim$ 0.8 & $R=16.23$ & $55.8"$\\
SDSSJ0819+5356 & 2 & 4.04 & $R=17.61$ & $<5''$ & SDSSJ1339+1310 & 2 & 1.69 & $R=16.37$ & $50.6''$\\
SDSSJ0904+1512 & 2 & 1.13 & $R=17.88$ & $<5''$ & SDSSJ1349+1227 & 2 & 3.00 & $R=17.59$ & $<5''$\\
SDSSJ0924+0219 & 4 & 1.78 & $R=15.75$ & $49.6''$ & SDSSJ1353+1138 & 2 & 1.41 & $R=16.74$ & $<5''$\\ 
SDSSJ1001+5027 & 2 & 2.86 & $R=17.69$ & $<5''$ & SDSSJ1400+3134 & 2 & 1.74 & $R=14.35$ & $58.8''$ \\
SDSSJ1002+4449 & 2 & 0.70 & $R=17.14$ & $20.2''$ & SDSSJ1406+6126 & 2 & 1.98 & $R=14.37$ & $45.7''$ \\ 
SDSSJ1054+2733 & 2 & 1.27 & $R=17.09$ & $<5''$ & SDSSJ1620+1203 & 2 & 2.77 & $R=13.97$ & $52.0''$\\
SDSSJ1131+1915 & 2 & 1.46 & $R=15.46$ & $58.2''$ & SDSSJ1650+4251 & 2 & 1.18 & $R=17.44$ & $<5''$\\
SDSSJ1313+5151 & 2 & 1.24 & $R=18.09$ & $<5''$ & SDSSJ2343$-$0050 & 2 & 1.51 & $R=16.65$ & $38.4''$\\
SDSSJ1322+1052 & 2 & 2.00 & $R=18.41$ & $<5''$ & & & & & &\\
   \hline
\end{tabular}
 \\
 \\
Table: Summary of our targets for the S11A Subaru/IRCS LGS-AO
observation. $N_{\rm img}$ indicates the number of quasar images, and
$\theta$ is the image separation between quasar images in units of
arcsec. ``mag'' is the magnitude of the tip-tilt star, ``dist''
is the angular distance of the tip-tilt star from the target. If dist
$<5''$ we use one of the quasar images as tip-tilt star.





%\begin{technicalinfo}
%\medskip
%\vspace{-0.5cm}
%We request reasonably deep IRCS$+$AO K' band (52 mas) imaging of $\sim30$ min or more, which including overhead will require under $\sim1$ h per target.

%In the Table we summarize the tip-tilt
%stars we will use. \\
%\begin{tabular}{lcccccclcccccc}
  % \hline\hline
%Name & $N_{\rm img}$ & $\theta$ & mag & dist & Name & $N_{\rm img}$ & $\theta$ & mag & dist  \\
   %  \hline
%SDSS J0743+2457 & 2 & 1.05 & $R=16.91$ & 23.4 & SDSS J1320+1644 & 2 & 8.58 & $R=16.55$ & $58.3$\\
%SDSS J0746+4403 & 2 & 1.08 & $R=15.45$ & 31.6 & SDSS J1322+1052 & 2 & 2.00 & $R=16.32$ & $29.7$\\
%SDSS J0819+5356 & 2 & 4.04 & $R=16.6$ & $35.4$ & SDSS J1335+0527 & 2 & $\sim$ 0.8 & $R=16.23$ & 55.8\\
%SDSS J0924+0219 & 4 & 1.78 & $R=15.75$ & 49.6 & SDSS J1339+1310 & 2 & 1.69 & $R=16.37$ & 50.6\\
%SDSS J0932+0722 & 2 & $\sim$ 2 & $R=14.33$ & 40.4 & SDSS J1353+1138 & 2 & 1.41 & $R=16.74$ & $\sim 0$\\ 
%SDSS J1001+5027 & 2 & 2.86 & $R=17.69$ & $\sim 0$ & SDSS J1400+3134 & 2 & 1.74 & $R=14.35$ & 58.8 \\
%SDSS J1002+4449 & 2 & 0.70 & $R=17.14$ & 20.2 & SDSS J1406+6126 & 2 & 1.98 & $R=14.37$ & 45.7 \\ 
%SDSS J1131+1915 & 2 & 1.46 & $R=15.46$ & 58.2 & SDSS J1620+1203 & 2 & 2.77 & $R=13.97$ & 52.0\\
%SDSS J1313+5151 & 2 & 1.24 & $R=16.28$ & $58.8$ & & & & & &\\
   %\hline
%\end{tabular}
% \\
 %\\
% Table: Summary of our targets for the S11A Subaru/IRCS LGS-AO
%observation. $N_{\rm img}$ indicates the number of quasar images, and
%$\theta$ is the image separation between quasar images in units of
%arcsec. ``mag'' is the magnitude of the tip-tilt star, ''dist''
%is the angular distance of the tip-tilt star from the target in units of arc sec. If dist
%$\sim 0$ we use one of the quasar images as tip-tilt star.





%\medskip
%\vspace{-0.5cm}
%We request reasonably deep IRCS$+$AO K' band (52 mas) imaging of $\sim30$ min or more, which including overhead will require under $\sim1$ h per target.

%In the Table we summarize the tip-tilt
%stars we will use. \\
%\begin{tabular}{lcccccclcccccc}
 %  \hline\hline
%Name & $N_{\rm img}$ & $\theta$ & mag & dist & Name & $N_{\rm img}$ & $\theta$ & mag & dist  \\
  %   \hline
%SDSS J0743+2457 & 2 & 1.05 & $R=16.91$ & 23.4 & SDSS J1320+1644 & 2 & 8.58 & $R=16.55$ & $58.3$\\
%SDSS J0746+4403 & 2 & 1.08 & $R=15.45$ & 31.6 & SDSS J1322+1052 & 2 & 2.00 & $R=16.32$ & $29.7$\\
%SDSS J0819+5356 & 2 & 4.04 & $R=16.6$ & $35.4$ & SDSS J1335+0527 & 2 & $\sim$ 0.8 & $R=16.23$ & 55.8\\
%SDSS J0924+0219 & 4 & 1.78 & $R=15.75$ & 49.6 & SDSS J1339+1310 & 2 & 1.69 & $R=16.37$ & 50.6\\
%SDSS J0932+0722 & 2 & $\sim$ 2 & $R=14.33$ & 40.4 & SDSS J1353+1138 & 2 & 1.41 & $R=16.74$ & $\sim 0$\\ 
%SDSS J1001+5027 & 2 & 2.86 & $R=17.69$ & $\sim 0$ & SDSS J1400+3134 & 2 & 1.74 & $R=14.35$ & 58.8 \\
%SDSS J1002+4449 & 2 & 0.70 & $R=17.14$ & 20.2 & SDSS J1406+6126 & 2 & 1.98 & $R=14.37$ & 45.7 \\ 
%SDSS J1131+1915 & 2 & 1.46 & $R=15.46$ & 58.2 & SDSS J1620+1203 & 2 & 2.77 & $R=13.97$ & 52.0\\
%SDSS J1313+5151 & 2 & 1.24 & $R=16.28$ & $58.8$ & & & & & &\\
  % \hline
%\end{tabular}
 %\\
% \\
% Table: Summary of our targets for the S11A Subaru/IRCS LGS-AO
%observation. $N_{\rm img}$ indicates the number of quasar images, and
%$\theta\ is the image separation between quasar images in units of
%arcsec. ''mag'' is the magnitude of the tip-tilt star, ''dist''
%is the angular distance of the tip-tilt star from the target in units of arcsec. If dist
%$\sim 0$ we use one of the quasar images as tip-tilt star.




%\medskip \vspace{-0.5cm} We request reasonably deep IRCS+AO188 K' band 20 mas imaging of 2h$\ \sim3$h or more, including overhead.  In the Table we summarize the tip-tilt stars which are accessible to use. 
%The cases where we use lensed quasar images themselves as tip-tilt stars are indicated by the distance $<5''$ in the Table. 
%\\    \begin{tabular}{lcccccc} \hline\hline Name & $N_{\rm img}$ & $\theta$ & $ z_s $ & $z_l$ & mag      & dist \\ \hline SDSS J1029+2623 & 3 & 22.5 & 2.198 & 0.6 & $R=18.58$ & $\sim0''$\\ & & & & & $R=18.46$ & $\sim22.5''$\\ RXJ1131-1231 & 4 & 3.80 & 0.66 & 0.29 & $R=15.81$ & $54.48''$\\ B1422+231 & 4 & 1.68 & 3.62 & 0.34 & $R=15.88$ & 56.76$''$\\  & & & & & $R=17.40$ & 17.47$''$\\ \hline \end{tabular} \\ \\ Table: Summary of our targets for the S12A Subaru/IRCS LGS-AO observation. $N_{\rm img}$ indicates the number of quasar images, $\theta$ is the image separation between quasar images in units of arcsec, and $z_s$ and $z_l$ are lens and source redshifts, respectively. ``mag'' is the magnitude of the tip-tilt star, ``dist'' is the angular distance of the tip-tilt star from the target. If dist $\sim0''$ we use one of the quasar images as tip-tilt star.


\end{technicalinfo}


%%%%%%%%%% 15. Condition of Closely-Related Past Observations %%%%%
% If your proposal is a continuation of (or inextricably 
% related with) the previously accepted proposals, describe 
% how the relevant past observations were carried out by giving 
% the Open Use proposal ID, Title, Weather/Observational condition, 
% Achievement rate [in %] for the planned outcome.
% Additional remarks: (a) The proposals described here must be 
% included also in the following "16 Post-Observational Status and 
% Publications". (b) The reason why you request observational time 
% in this semester (in spite of the past experience of Subaru 
% observations on a similar/the same theme) has to be briefly 
% described at the end of the "Scientific Justification" (e.g., 
% bad weather/condition, telescope/instrument down time , 
% expansion/improvement of the data, observing targets in 
% different season). 
\begin{relationto}
\relatobs{}{}{}{}
\end{relationto}

%%%%%%%%%% 16. Post-Observation Status and Publications %%%%%%%%%%
% Report the status or outcome of your main Subaru Observations 
% carried out in the past. All observations relevant to this 
% proposal (e.g., those enumerated in 15. above) should be included 
% here. Similarly, all observations in these 3 years with which 
% you were involved as P.I. must be reported. 
% Give the date, the Open Use proposal ID (e.g., S01B-999), 
% PI's name, status of data reduction/analysis, and related 
% publications. 
\begin{previoususe}
\pastrun{2010/3}{S10A-061}{Iye, M}{in progress}{papers in prep.}
\pastrun{2009/4}{S09A-103}{Oguri, M}{in progress}{papers in prep.}
\pastrun{2009/2,4}{S09A-065}{Iye, M.}{Finished}{papers in prep.}
\pastrun{2009/1}{S08B-027}{Inada, N}{Finished}{2 papers (AJ)}
\pastrun{2008/10}{S08B-019}{Iye, M.}{Finished}{Ota et al. (2009) IAP Conf.}
\pastrun{2007/6}{S07A-082}{Oguri, M}{Finished}{3 papers (ApJ$\times2$/MNRAS)}
%\pastrun{2007/5}{S07A-059}{Ota, K.}{Clouded out}{no result}
\pastrun{2007/4}{S07A-102}{Iye, M.}{Finished}{Ota et al. (2008) ApJ}
\pastrun{2007/1}{S06B-103}{Inada, N}{Finished}{3 papers (AJ$\times$2, PASJ)}
\pastrun{2006/4}{S06A-062}{M.Iye}{Finished}{Iye etal(2006) Nature 443, 186}


% above line must be kept blank.
%%%%%%%%%% 17. Thesis Work %%%%%%%%%%
% If these observations are intended to be part of a student's 
% thesis, enter the student's name and thesis title.
\thesis{Cristian Eduard Rusu}{LGS AO observations of SDSS Gravitationally Lensed Quasars}

% above line must be kept blank.
%%%%%%%%%% 18. Subaru Open Use Intensive Program %%%%%%%%%%
% Uncomment this line if this is an Open Use Intensive Program
%\intensive


\end{previoususe}

%%%%%%%%%% 19. List of (more) Targets  %%%%%%%%%%
%*** (optional: if the space in 9. is insufficient) ***%
% If you have more targets, please uncomment the following 
% 3 lines and enter them here, using the same format as 
% in entry 9.
\begin{moretargets}
       \target{SDSSJ1335+0527}{13 35 12.11}{+05 27 32.4}{J2000}{SDSS $r=19.60$}
       \target{SDSSJ1339+1310}{13 39 07.14}{+13 10:39.6}{J2000}{SDSS $r=18.71$}  
       \target{SDSSJ1349+1227}{13 49 29.84}{+12 27 06.3}{J2000}{SDSS $r=17.79$}     
       \target{SDSSJ1353+1138}{13 53 06.35}{+11 38 04.7}{J2000}{SDSS $r=16.46$}
       \target{SDSSJ1400+3134}{14 00 12.77}{+31 34:54.1}{J2000}{SDSS $r=19.84$}
       \target{SDSSJ1406+6126}{14 06 24.83}{+61 26:40.9}{J2000}{SDSS $r=19.28$}  
    \target{SDSSJ1620+1203}{16 20 26.14}{+12 03 42.0}{J2000}{SDSS $r=19.27$}
\target{SDSSJ1650+4251}{16 50 43.44}{+42 51 49.3}{J2000}{SDSS $r=17.41$}
\target{SDSSJ2343$-$0050}{23 43 11.94}{-00 50 34.3}{J2000}{SDSS $r=19.59$}

\end{moretargets}

%%%%%%%%%% 20. List of Backup Targets %%%%%%%%%%
%*** (optional) ***%
% If you have backup targets decribed in entry 13, 
% please uncomment the following 3 lines and enter their data here, 
% using the same format as in entry 9. 
%\begin{backuptargets}
%\target{}{}{}{}{}
%\end{backuptargets}

%%%%%%%%%% Scientific Justification %%%%%%%%%
% The Scientific Justification (SJ) of the proposal
% should be prepared in a separate PDF file with its size
% no larger than 2 MB.
% Please be sure to strictly keep to the requirements of SJ 
% described in Sect. 2 of
%http://subarutelescope.org/Observing/Proposals/Submit/howto.html

\end{document}
